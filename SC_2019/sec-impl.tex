

\section{Implementation}
\label{sec:impl}
%
This section discusses implementing relational algebra efficiently and in parallel. We take a hybrid approach of nesting key-value stores within a hash-table that can be partitioned across multiple cores or nodes. 


\subsection{Representations}

First, we review the two main approaches to encoding relations in a way that is amenable to fast RA algorithms. Unsurprisingly, algorithms for computing join, union, selection, etc, depend greatly on the representation used for relations themselves.

\paragraph{Decision diagrams} Decision diagrams such as binary decision diagrams (BDDs) and zero-supressed binary decision diagrams (ZDDs) are compact representations of relations as decision trees. Each variable (column) storing an integer is broken down into one variable per bit---a relation R(a,b,c) where each column stores a 64bit integer is encoded as a set of binary strings, each 192 bits long.  


\paragraph{Key-value stores} Another approach to encoding relations is to use a hash table, B-tree, prefix tree (trie), or other key-value store.


\paragraph{Hybrid hash-table and b-tree} Our approach is to use an efficient key-value store, but to



\subsection{Hybrid Join}

...

\subsection{Distributed Join}

...
